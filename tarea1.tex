\documentclass{article}

\usepackage[margin = 1cm]{geometry}

\begin{document}

%Titulo
\title{Autómatas y Lenguajes formales 2019-2\\
        \large Ejercicio Semanal 1}

\date{1 de febrero del 2019}

\author{Sandra del Mar Soto Corderi\\
        Edgar Quiroz Castañeda}

\maketitle

%Ejercicios

\begin{enumerate}
    %1
    \item {
        Demuestre las siguientes propiedades de las operaciones sobre cadenas 
        vistas en clase
        
        \begin{itemize}
            \item {
                Identidad $v\epsilon = \epsilon v = v$.

                Por como está definida la concatenación, se tiene que $v \epsilon = v$.\\
                Entonces sólo falta verificar que $\epsilon v = v$.

                Por inducción sobre $v$.
                \begin{itemize}
                    \item {
                        Casp base $v = \epsilon$\\
                        Entonces $\epsilon v = \epsilon\epsilon$\\
                        Y por la definición de concatenación, 
                        $\epsilon\epsilon = \epsilon = v$.\\
                        Por lo tanto $\epsilon v = v$
                    }

                    \item {
                        Paso inductivo, con hipótesis $\epsilon v = v$.\\ 
                        Hay que demostrar que $\epsilon (va) = va$, con $a$ un símbolo del alfabeto de $v$.\\
                        Por la definición de la concatenación, tenemos que $\epsilon (va) = (\epsilon v)a$.\\
                        Y usando la hipótesis, $ = va$.\\
                        Por lo que $\epsilon (va) = va$
                    }
                \end{itemize}
                Por lo tanto, $v \epsilon = v = \epsilon v$
            }

            \item {
                Longitud $|vw| = |v| + |w|$.

                Por inducción sobre $w$.

                \begin{itemize}
                    \item {
                        Caso base $w = \epsilon$\\
                        Entonces $vw = v\epsilon = v$\\
                        Y por la definición de longitud
                        \[|vw| = |v| = |v| + 0 = |v| + |\epsilon| = |v| + |w|\]
                    }

                    \item {
                        Paso inductivo, con hipótesis $|vw| = |v| + |w|$.\\ 
                        Hay que demostrar que $|v(wa)| = |v| + |wa|$, con $a$ 
                        un símbolo del alfabeto de $v$.\\
                        Como la concatenación asocia (demostrado en clase), 
                        tenemos que $v(wa) = (vw)a$.\\
                        Y por la definición longitud, tenemos que
                        $|(vw)a| = |vw| + 1$.\\
                        Y usando la hipótesis $|vw| + 1 = |v| + |w| + 1$.\\
                        Luego, notemos que $|wa| = |w| + 1$, por la definición de longitud.\\
                        Por lo que $|v(wa)| = |v| + |w| + 1 = |v| + |wa|$.
                    }
                \end{itemize}
                Por lo tanto, se cumple que $|vw| = |v| + |w|$.
            }

            \item {
                Reversa $(v^R)^R = v$

                Por inducción sobre $v$
                \begin{itemize}
                    \item {
                        Caso base $v = \epsilon$.\\
                        Entonces $(v^R)^R = (\epsilon^R)^R = \epsilon ^ R = 
                        \epsilon = v$, por la definición de reversa.       
                    }

                    \item {
                        Paso inductivo, con hipótesis $(v^R)^R = v$.\\
                        Hay que demostrar que $((va)^R)^R = va$, con $a$ un 
                        símbolo del alfabeto de $v$.\\
                        Por definición de reversa, $(va)^R = av^R$.\\
                        Luego, utilizando la propiedad de $(uw)^R = w^Ru^R$ 
                        (demostrada en clase), $(av^R)^R = (v^R)^Ra^R$.\\
                        Y usando la hipótesis, $(v^R)^Ra^R = va^R$.\\
                        Luego, notemos que $a = \epsilon a$, por la propiedad 
                        de identidad.\\
                        Por lo que $a^R = (\epsilon a)^R = 
                        a  \epsilon^R = a \epsilon = a$.\\
                        Entonces $va^R = va$, por lo que $((va)^R)^R = va$.
                    }
                \end{itemize}
                Por lo tanto, siempre se cumple que $(v^R)^R = v$.
            }
        \end{itemize}
    }

    %2
    \item {
        Da todos los prefijos de las siguientes cadenas

        \begin{itemize}
            \item {
                1001 \\
                Los prefijos son 
                \begin{itemize}
                    \item {
                        1001, pues $1001 = 1001 \cdot \epsilon$ 
                    }
                    \item {
                        100, pues $1001 = 100 \cdot 1$ 
                    }
                    \item {
                        10, pues $1001 = 10 \cdot 01$ 
                    }
                    \item {
                        1, pues $1001 = 1 \cdot 001$ 
                    }
                    \item {
                        $\epsilon$, pues $1001 = \epsilon \cdot 1001$ 
                    }
                \end{itemize}
            }

            \item {
                aaabbb \\
                Los prefijos son 
                \begin{itemize}
                    \item {
                        aaabbb, pues aaabbb = aaabbb $\cdot$ $\epsilon$
                    }
                    \item {
                        aaabb, pues aaabbb = aaabb $\cdot$ b
                    }
                    \item {
                        aaab, pues aaabbb = aaab $\cdot$ bb 
                    }
                    \item {
                        aaa, pues aaabbb = aaa $\cdot$ bbb
                    }
                    \item {
                        aa, pues aaabbb = aa $\cdot$ abbb
                    }
                    \item {
                        a, pues aaabbb = a $\cdot$ aabbb
                    }
                    \item {
                        $\epsilon$, pues aaabbb = $\epsilon$ $\cdot$ aaabbb
                    }
                \end{itemize}
            }
        \end{itemize}
    }

    %3
    \item {
        Da todos los sufijos de las siguientes cadenas

        \begin{itemize}
            \item {
                1010 \\
                Los sufijos son 
                \begin{itemize}
                    \item {
                        1010, pues $1010 = \epsilon \cdot 1010$ 
                    }
                    \item {
                        010, pues $1010 = 1 \cdot 010$ 
                    }
                    \item {
                        10, pues $1010 = 10 \cdot 10$ 
                    }
                    \item {
                        0, pues $1010 = 101 \cdot 0$ 
                    }
                    \item {
                        $\epsilon$, pues $1010 = 1010 \cdot \epsilon$ 
                    }
                \end{itemize}
            }

            \item {
                abbabba \\
                Los sufijos son
                \begin{itemize}
                    \item {
                        abbabba, pues abbabba = $\epsilon \cdot$ abbabba
                    }
                    \item {
                        bbabba, pues abbabba = a $\cdot$ bbabba
                    }
                    \item {
                        babba, pues abbabba = ab $\cdot$ babba
                    }
                    \item {
                        abba, pues abbabba = abb $\cdot$ abba
                    }
                    \item {
                        bba, pues abbabba = abba $\cdot$ bba 
                    }
                    \item {
                        ba, pues abbabba = abbab $\cdot$ ba
                    }
                    \item {
                        a, pues abbabba = abbabb $\cdot$ a 
                    }
                    \item {
                        $\epsilon$, pues abbabba = abbabba $\cdot \epsilon$
                    }
                \end{itemize}
            }
        \end{itemize}
    }

    %4
    \item {
        ¿Porqué $\epsilon$ es subcadena de cualquier cadena?\\
        Sea $v$ aquella subcadena cualquiera. \\
        Toda subcadena $w$ de $v$ cumple que $v = x_1wx_2$ para algunas 
        $x_1, x_2 \in \Sigma^*$. \\
        Así que con $x_1 = v, x_2 = \epsilon$, tenemos que 
        $v = v \epsilon = v \epsilon \epsilon = x_1 \epsilon x_2$, por la propiedad de identidad de $\epsilon$.\\
        Por lo tanto, $\epsilon$ es subcadena de cualquier cadena.
    }

\end{enumerate}
    
\end{document}